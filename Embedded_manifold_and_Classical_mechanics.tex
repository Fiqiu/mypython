\documentclass[13pt]{article}
\usepackage[utf8]{inputenc}	% Para caracteres en español
\usepackage{amsmath,amsthm,amsfonts,amssymb,amscd}
\usepackage{multirow,booktabs}
\usepackage[table]{xcolor}
\usepackage{fullpage}
\usepackage{lastpage}
\usepackage{enumitem}
\usepackage{fancyhdr}
\usepackage{mathrsfs}
\usepackage{wrapfig}
\usepackage{setspace}
\usepackage{calc}
\usepackage{multicol}
\usepackage{cancel}
\usepackage[retainorgcmds]{IEEEtrantools}
\usepackage[margin=3cm]{geometry}
\usepackage{amsmath}
\newlength{\tabcont}
\setlength{\parindent}{0.0in}
\setlength{\parskip}{0.05in}
\usepackage{empheq}
\usepackage{framed}
\usepackage[most]{tcolorbox}
\usepackage{xcolor}
\colorlet{shadecolor}{orange!15}
\parindent 0in
\parskip 12pt
\geometry{margin=1in, headsep=0.25in}
\theoremstyle{definition}
\newtheorem{defn}{Definition}
\newtheorem{reg}{Rule}
\newtheorem{exer}{Exercise}
\newtheorem{note}{Note}
\newcommand{\bm}{\textbf}
\begin{document}

\title{Embedded Manifold and Classical Mechanics}

\thispagestyle{empty}

\begin{center}
{\LARGE \bf Embedded Manifolds and Classical Mechanics}\\
{\large Junpeng Gao}\\
junpenggaox@gmail.com
\end{center}
\section{Definition of Manifolds}
\subsection{Linear Algebra}

The cross product, viewed as an operator on $\mathbb{R}^{3},$ can also be written in terms of a linear transformation and a matrix multiplication as
$$
x \times y=S(x) y
$$

$\text { where } S: \mathbb{R}^{3} \rightarrow \mathbb{R}^{3 \times 3} \text { is }$

\begin{equation}
S(x)=\left[\begin{array}{ccc}
0 & -x_{3} & x_{2} \\
x_{3} & 0 & -x_{1} \\
-x_{2} & x_{1} & 0
\end{array}\right]
\end{equation}
Note that $\operatorname{det}(S(x))=0,$ for all $x \in \mathbb{R}^{3}$.

For each orthogonal matrix $R \in \mathrm{SO}(3),$ there is a skew-symmetric matrix $\xi \in \mathfrak{s o}(3)$ such that $I_{3 \times 3}+\xi$ is nonsingular and the orthogonal matrix can be expressed as:
\begin{equation}
R=\left(I_{3 \times 3}-\xi\right)\left(I_{3 \times 3}+\xi\right)^{-1}
\end{equation}
This relationship is often referred to as the Cayley transformation.

The following is also true. For each orthogonal matrix $R \in \mathrm{SO}(3),$ there is a skew-symmetric matrix $\xi \in \mathfrak{s o}(3)$ such that the orthogonal matrix can be expressed as:
\begin{equation}
R=e^{\xi}=\sum_{n=0}^{\infty} \frac{(\xi)^{n}}{n !}
\end{equation}
\begin{equation}
\left(\frac{\partial f(A)}{\partial A} \cdot B\right)=\operatorname{trace}\left(\left(\frac{\partial f(A)}{\partial A}\right)^{T} B\right) \in \mathbb{R}^{1}
\end{equation}

\subsection{Manifolds}
\bm{Several important concepts and definitions:}\\

A point in a differentiable manifold $\mathcal{M}$ is most often described in terms of
local coordinates, called charts, and a collection of charts, called an atlas.
Each vector in $\mathcal{M}$ is specified by real-valued coordinates.

\bm{Manifold:}

A differentiable manifold, as a submersion or embedded manifold in $\mathbb{R}^n$,
is described by
\begin{equation}
M=\left\{x \in \mathbb{R}^{n}: f_{i}(x)=0, i=1, \dots, l\right\}
\end{equation}
where $f_{i}: \mathbb{R}^{n} \rightarrow \mathbb{R}^{1}, i=1, \ldots, l$ are scalar differentiable functions with the property that the vectors $\frac{\partial f_{i}(x)}{\partial x}, i=1, \ldots, l$ are linearly independent vectors in $\mathbb{R}^{n}$ for each $x \in M .$ Thus, necessarily $1 \leq l \leq n .$ We say that the manifold
M has dimension $n − l$ and codimension $l$. Consequently, we can represent a vector in an embedded manifold $M \subset \mathbb{R}^{n}$ as a vector in $\mathbb{R}^{n}$ so long as the equality conditions defining the embedded manifold are satisfied.

\bm{Tangent Space:}

Let $\gamma:[-1,1] \rightarrow M$ denote a differentiable curve on $M$ with $\gamma(0)=x \in M$ Then $\frac{d \gamma(s)}{d s},$ evaluated at $s=0,$ is a tangent vector to $M$ at $x \in M .$ For each $x \in M,$ the set of all such tangent vectors to $M$ at $x \in M,$ denoted by $\mathrm{T}_{x} M$ is a subspace of $\mathbb{R}^{n},$ referred to as the tangent space of $M$ at $x \in M,$ and hence it is a linear manifold. We refer to $\xi \in \mathrm{T}_{x} M$ as a tangent to $M$ at $x \in M .$ For a manifold $M$ as given above, it can be shown that the tangent space.
\begin{equation}
\mathrm{T}_{x} M=\left\{\xi \in \mathbb{R}^{n}:\left(\frac{\partial f_{i}(x)}{\partial x} \cdot \xi\right)=0, i=1, \ldots, m\right\}
\end{equation}
so that the tangent space consists of the set of vectors in $\mathbb{R}^{n}$ that are orthogonal to all of the gradients of the functions that define the manifold. The dimension of the tangent space is $n-m$.

\bm{Tangent bundles}:

The tangent bundle of a manifold $M,$ denoted by $\mathrm{T} M,$ is the set of pairs $(x, \xi) \in M \times \mathrm{T}_{x} M,$ and it has its own manifold structure. The dimension of the tangent bundle is $2(n-m)$.

\bm{Cotangent Vector space and bundles:}

The set of all such cotangents, denoted $\mathrm{T}_{x}^{*} M,$ is the set of all Binear functionals on the tangent space at $x \in M,$ namely
\begin{equation}
\mathrm{T}_{x}^{*} M=\left(\mathrm{T}_{x} M\right)^{*}
\end{equation}
and it is a subspace, referred to as the cotangent space of $M$ at $x \in M,$ and hence it is a linear manifold. The dimension of the cotangent space is $n-m$.

The cotangent bundle of a manifold $M,$ denoted by $T^{*} M,$ is the set of pairs $(x, \zeta) \in M \times \mathrm{T}_{x}^{*} M,$ and it has its own manifold structure. The dimension of the cotangent bundle is $2(n-m)$.
\subsection{Vector Fields on a Manifold}
\bm{Vector Fields:}

Let $M$ be a differentiable manifold embedded in $\mathbb{R}^{n} .$ A vector field on $M$ is defined by a mapping from $M$ to $\mathbb{R}^{n}$ with the property that for each $x \in M$ there is a unique $y \in \mathrm{T}_{x} M$ and this correspondence $x \rightarrow y$ is continuous. The interpretation is that a vector field associates with each point $x \in M$ in the manifold a unique tangent vector $y \in \mathrm{T}_{x} M$ in the tangent space of the manifold.

\bm{Differential Equations:}

Let $f: M \times \mathbb{R}^{1} \rightarrow \mathbb{R}^{n}$ be a differentiable time-dependent vector-valued function that satisfies $f(x, t) \in \mathrm{T}_{x} M$ for each $x \in M$ and $t \in \mathbb{R}^{1} .$ If the vector field $f(x, t)$ is time-independent, then it is said to be autonomous, otherwise it is nonautonomous. The differential equation
\begin{equation}
\dot{x}(t)=f(x(t), t)
\end{equation}
on the manifold $M$ is well-posed in the sense that for each initial condition in the manifold $M,$ the initial-value problem has a unique solution. Suppose 
a solution at time instant $t,$ corresponding to an initial-value $x_{0} \in M$ at time instant $t_{0},$ is given by $x\left(t, t_{0}, x_{0}\right) \in M .$ Then $x\left(t, t_{0}, x_{0}\right)=F_{t, t_{0}}\left(x_{0}\right) \in M$
denotes the one-parameter time evolution or motion on $M .$ The operator $F_{t, t_{0}}: M \rightarrow M$ defines the flow map.  

If $\phi: M \rightarrow \mathbb{R}^{1}$ is differentiable, then we can comp time derivative
\begin{equation}
\dot{\phi}(x)=\frac{\partial \phi(x)}{\partial x} \cdot f(x)
\end{equation}

\begin{shaded}

\end{shaded}


\end{document}